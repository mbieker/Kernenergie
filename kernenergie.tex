\documentclass[11pt,a4paper]{article}

\usepackage[utf8]{inputenc}
\title{\textbf{Grundlagen der Reaktorphysik}\\Gliederung}

\date{}
\begin{document}

\maketitle
\section{Kernspaltung}
\subsection{Beschreibung der Allgemeinen Reaktion}
\[
^A_Z X + ^1_0n \rightarrow ^{A_1}_{Z_1}Y_1 +  ^{A_2}_{Z_2}Y_2 + \nu\cdot ^1_0n
\] 
\subsection{Übersicht verschiedener Spaltstoffe}
\begin{itemize}
\item Uran-235 (natürlich vorkommend im Uranerz mit 0.714 Gew.-Prozent)
\item Plutoium-241
\item Plutonium-239
\item Uran-233
 \end{itemize}
Unterscheidung von natürlich vorkommenden Spaltstoffen ( U-235) und durch andere Kernreaktionen erbrütete Stoffe.
\subsection{Wirkungsquerschnitte}
Einführung von $\sigma_f$ und $\sigma_a$ als Wirkungsquerschnitte für Kernspaltung und (parasitäre Absorption). 

\subsection{Berechnung der Reaktionsenergie am Beispiel einer Spaltung von U-235}
\[
^{235}_{92} U + ^1_0n \rightarrow ^{143}_{56}Ba +  ^{90}_{36}Kr + 3 \cdot ^1_0n
\]
\begin{itemize}
\item Aufstellen der Massenbilanz vor und nach der Reaktion (Tabelle)
\item Bestimmung des Massendefekts $\Delta m$.
\item Einsteinsche Energie-Masse Beziehung : $E_f = \Delta m c^2$
\item Berechnung der Reaktionsenergie pro Gramm ( Energiedichte)
\item Vergleich mit den Energiedichten konventioneller Energieträger (Kohle, Gas, etc.) 
$\rightarrow $ Kernenergie vs. "Atomenergie"
\end{itemize}
\subsection{Spaltprodukte und Nachzerfallswärme}
Die Produkte der Kernspaltung sind sehr oft ebenfalls radioaktiv und werden erst durch nachfolgende Zerfälle (meist $\beta^-$ und $\gamma$) in stabile Kerne umgewandelt. Für den Reaktorbau zwei Aspekte:
\begin{itemize}
\item Einige Spaltprodukte sind sehr starke Neutronenabsorber \\$\rightarrow$ Einfluss auf die Neutronenbilanz
\item Durch den Zerfall der Spaltprodukte wird auch nach der Beendigung der Kettenreaktion  Energie freigesetzt (Nachzerfallswärme) \\$\rightarrow$ Kühlung muss gewährleistet werden.$\rightarrow$ Sicherheitsrisiko!
\end{itemize}
\subsection{Neutronenfreisetzung der Kernspaltung}
Ohne die Freisetzung mehrerer Neutronen bei der Kernspaltung wäre keine Kettenreaktion möglich. Nur so kann die Kernspaltung als kontinuierlich arbeitende Energiequelle eingesetzt werden. Es entstehen auf zwei Arten Neutronen:
\begin{itemize}
\item Prompte Neutronen: Werden innerhalb von $ 10^{-4}\,s $ freigesetzt.
\item Verzögerte Neutronen: Entstehen durch den Zerfall von Zwischenprodukten der Kernreaktion.
\end{itemize}
Letztere sind auf Grund ihrer langen Lebensdauer für die Neutronendynamik im Reaktor wichtig.
\section{Neutronenphysik}
\subsection{Neutronenfluss und Reaktionsraten}
Definition des Neutronenflusses 
\[
\Phi(E) = \rho_n(E) \cdot v(E).
\]
Bei gegebenem Neutronenfluss können so Reaktionsraten bestimmt werden.
\[
R = n\cdot v \cdot \sigma \cdot N = \Phi \cdot \Sigma 
\]
\subsection{Wechselwirkugen}
Die Neutronen treten auf verschiedene Arten mit dem Reaktormedium in Wechselwirkung.
\begin{itemize}
\item Streuung
\item (n, $\alpha$) -Reaktion
\item (n, p ) -Reaktion
\item (n, $\gamma$) -Reaktion
\item (n , 2n) -Reaktion
\item Kernspaltung
\end{itemize}
Einige dieser Wechselwirkungen werden als parasitär bezeichnet, da diese Neutronen absorbieren, welche dann nicht mehr für die Kettenreaktion zur Verfügung stehen.
\subsection{Energiespektren}
Sehr viele Wechselwirkungen hängen von der Energie der Neutronen ab. So sind zum Beispiel die Wirkungsquerschnitte für die Kernspaltung stark energieabhängig. Einteilung der Neutronen auf Grund ihrer  Energie in
\begin{itemize}
\item (schnelle) Spaltneutronen
\item Epithermische Neutronen
\item Thermische Neutronen.
\end{itemize}
\subsection{Diffusion und Abbremsung}
\begin{itemize}
\item Beschreibung der Bewegung der Neutronen im Reaktormedium (evtl. Diffusionsgleichung)
\item Abbremsung der Neutronen auf thermische Energien durch den Moderator.
\end{itemize}
\section{Kettenreaktion}
\subsection{Stabilität von Kettenreaktionen}
Unterscheidung von 
\begin{itemize}
\item kontrollierte Kettenreaktion ($\rightarrow$ Kernreaktor)
\item unkontrollierte Kettenreaktion ($\rightarrow$ Kernwaffe)
\end{itemize}
Notwendig für sicheren Reaktorbetrieb: Kriterium für die Stabilität der Kettenreaktion\\ $\rightarrow$ Multiplikationsfaktor.
\subsection{Vierfaktorformel}
\[
k_\infty = \epsilon \cdot p \cdot f \cdot \eta\ ( = 1)
\]
Herleitung mit Erläuterung der einzelnen Faktoren
\begin{itemize}
\item $\epsilon$: Schnellspaltfaktor
\item $p$: Resonanzentkommwahrscheinlichkeit
\item $f$: Thermische Nutzung
\item $\eta$: Neutronenausbeute
\end{itemize}
Einführung von Korrekturtermen für endliche Reaktoren. \\ Möglichkeit der Reaktorsteuerung durch Veränderung der Faktoren (Steuerstäbe, Bohrsäure, etc.)
\subsection{Kritische Reaktoren}
Bestimmung der kritischen Masse für verschiedene Reaktorgeometrien
\end{document}
