\documentclass[11pt,a4paper]{article}
\usepackage[utf8]{inputenc}
%Gummi|065|=)
\title{\textbf{V353: Relaxationsverhalten eines RC-Kreises}\\Kollog-Fagren}
\date{}

\begin{document}

\maketitle

\section{Theorie}
\begin{itemize}
\item Was ist Relaxation?
\item Skizze eines RC-Gliedes?
\item Welche Form hat die Entladekurve eines Kondensators?
\item Welche Form haben die Differentialgleichungen die den RC-Kreis beschreiben?
\item Wie hängt die Kondensatorspannung von der Anregungsfrequenz ab?
\item Wie hängt die Phasendifferenz von Kondensator- und Generatorspannung von der Anregungsfrequenz ab?
\item Welche Bedingung muss an die Frequenz eines Signals gestellt werden, damit ein RC-Glied als Integrator verwendet werden kann?
\end{itemize}
\section{Messprogramm}
\subsection{Entladungskurve des RC Kreises}
\begin{itemize}
\item 20 (U,t) - Messwerte
\end{itemize}

\subsection{Kondensatorspannung in Abhängigkeit von der Generatorspannung}
\begin{itemize}
\item 15-20 (U,f) - Messwerte von 10 Hz bis 15KHz
\end{itemize}
\subsection{Phasenverschiebung der Kondensatorspannung in Abhängigkeit von der Generatorspannung}
\begin{itemize}
\item 15-20 (a,b,f) - Messwerte von 10 Hz bis 15KHz
\end{itemize}
\subsection{RC-Kreis als Integrator}
\begin{itemize}
\item Erstellung von drei Screenshots verschiedener Spannungsformen zum Nachweis der Integratoreigenschaft des RC-Gliedes.
\end{itemize}
\section{Durchführung}
\subsection{Entladungskurve des RC Kreises}
\begin{itemize}
\item Skizze der Schaltung
\item Welche Spannungsform und Frequenz muss am Generator eingestellt werden, um die Entladungskurve auf dem Oszilloskop sichtbar zu machen?
\item Wie muss getriggert werden?
\item Wie muss das Signal auf das Oszilloskop gekoppelt werden (AC oder DC)?
\end{itemize}
\subsection{Messung von Kondensatorspannung und Phasenverschiebung in Abhängigkeit von der Anregungsfrequenz}
\begin{itemize}
\item Skizze der Schaltung.
\item Welche Spannungsform wird am Generator eingestellt?
\item Wie müssen in den verschiedenen Frequenzbereichen die Signale an das Oszilloskop gekoppelt werden (AC oder DC)?
\item Warum müssen sowohl Generatorspannung als auch die Kondensatorspannung gemessen werden?
\item Die Amplituden werden mit der Measure-Funktion des Oszilloskops gemessen.
\item Wie kann aus den beiden Signalen die Phasenverschiebung bestimmt werden (Skizze)?
\end{itemize}
\subsection{Der RC-Kreis als Integrator}
Es sind bei einer Frequenz $\gg 5$ kHz Screenshots von drei verschiedenen Spannungen anzufertigen.
\end{document}
